\newpage 
\section{Eksamensopgave 1 - Lys}\label{sec:Exam01}
\subsection{God viden til afsnit: \ref{sec:Exam01}}
\subsubsection{Fysiskestørelser}
\begin{center}
    \renewcommand{\arraystretch}{1.5}
    \begin{tabular*}{\textwidth}{@{\extracolsep{\fill}} c l c}
        \hline
        Symbol & Navn af størrelse & Enhed \\
        \hline
        \begin{math}v\end{math} & bølgens udbredelseshastighed & \begin{math}m s^{-1}\end{math}\\
        \begin{math}\lambda\end{math} & bølgelængde & m \\
        \begin{math}f\end{math} & frekvens & Hz \\
        \begin{math}T\end{math} & periode & s \\
        \begin{math}I\end{math} & indfaldsvinkel & grader \\
        \begin{math}B\end{math} & brydningsvinkel & grader \\
        \begin{math}I_c\end{math} & kritisk vinkel & grader \\
        \begin{math}n_1\end{math} & brydningsindex 1 & 1 \\
        \begin{math}n_2\end{math} & brydningsindex 2 & 2 \\
        \hline
    \end{tabular*}
\end{center}
\subsection{Beregn lysets frekvens i luft når bølgelængden er 560nm}
For at finde frem til frekvensen ved en bølgelængde på 560 nm må følgende formel anvendes:
\begin{equation*}
    v=\frac{\lambda}{T}
\end{equation*}
Vi ved, at perioden\begin{math}T\end{math} og frekvensen \begin{math}f\end{math} er relateret ved:
\begin{equation*}
    f=\frac{1}{T}
\end{equation*}
Dette betyder, at frekvensen er det inverse af perioden. Hvis perioden (\begin{math}T\end{math}) er lang, vil frekvensen (\begin{math}f\end{math}) være lav, og omvendt.\\\\
Hvis \begin{math}T\end{math} erstattes med \begin{math}\frac{1}{f}\end{math}
\begin{equation*}
    v=\frac{\lambda}{\frac{1}{f}}
\end{equation*}
For at få \begin{math}f\end{math} til at stå alene ganges der nu med \begin{math}1/f\end{math} på begge sider af lighedstegnet. 
\begin{equation*}
    v=\lambda \cdot f
\end{equation*}
For at finde frekvensen isoleres f på den ene side af lighedstegnet. Dette gøres ved at dividere med bølgelængden.
\begin{equation*}
    f=\frac{v}{\lambda}
\end{equation*}
Herefter indsættes værdierne i formlen og regnes ud. Det er opgivet at lysets hastighed er \begin{math}3,00 10^{8} m/s\end{math}. Og at lysets bølgelængde er 560 nm. Hvilket svarer til 560 x 10\textsuperscript{-9} m.
\begin{equation*}
    f=\frac{3,00 \cdot 10^{8} m/s}{560 \cdot 10^{-9} m}
\end{equation*}
Da m går ud med m så har man kun tilbage at:
\begin{equation*}
    f=5,36 \cdot 10^{14} s^{-1}
\end{equation*}
For at omskrive \begin{math}s^{-1}\end{math} til Hz anvender man det faktum at 1 Hz = 1 s\textsuperscript{-1}.
\textbf{\begin{equation*}
    \mathcolorbox{yellow}{f=5,36 \cdot 10^{14} Hz}
\end{equation*}}
% TODO: vurder om dette afsnit skal stå i teksen. En ting der er værd at vide er at \begin{math}v\end{math} også kendt som \begin{math}v_b\end{math} er bølgens udbredelseshastighed. Bølgens udbredelseshastighed er den hastighed, hvormed en bølge bevæger sig gennem et medium. 
\subsection{Beregn brydningsvinklen når indfaldsvinklen er 15. Lav en skite der ilustrerer brydningen}
For at finde frem til brydningsvinklen skal man bruge snells lov. Snells lov er en lov der beskriver hvordan lys brydes når det går fra et matriale til et andet. Brydningen skyldes hastighedsforskellen i de to materialer. Her er I indfaldsvinklen og B brydningsvinklen. 
\begin{equation*}
    \frac{sin(I)}{sin(B)}=\frac{n_2}{n_1}
\end{equation*}
Da man ønsker at finde brydningsvinklen ved 15 grader, skal man reducere udtrykket så man kan finden det rigtige udtryk for brydningsvinklen. Det gør man ved at isolere \begin{math}sin(B)\end{math} på den ene side af lighedstegnet. Det gøres ved at gange \begin{math}sin(B)\end{math} over på den anden side af lighedstegnet. 
\begin{equation*}
    sin(I)=sin(B) \cdot \frac{n_2}{n_1}
\end{equation*}
Herefter dividres med \begin{math}\frac{n_2}{n_1}\end{math} for at isolere \begin{math}sin(B)\end{math}
Så kommer det tilsidst til at se sådan her ud:
\begin{equation*}
    sin(B)=sin(I)\cdot\frac{n_1}{n_2}
\end{equation*}
Så indsættes værdierne i formlen og regnes ud. 
\begin{equation*}
    sin(B)=sin(15)\cdot\frac{1}{1.33}
\end{equation*}
\begin{equation*}
    sin(B)=0,1946
\end{equation*}
Så bruges invers sinus for at finde vinklen.
\begin{equation*}
    B=sin^{-1}(0,1946)
\end{equation*}
Dette giver en brydningsvinkel på 11 grader.
\begin{equation*}
    \mathcolorbox{yellow}{B=11^{\circ}}
\end{equation*}
\subsection{Find den kritiske vinkel hvor der totalreflektionen indtræffer}
Et nyt forsøg laves hvor lysstrålen sendes fra sprit op i luften. 
For at finde den kritiske vinkel skal man bruge følgende formel:
\begin{equation*}
    sin(I_c)=\frac{n_2}{n_1}
\end{equation*}
Her er \begin{math}I_c\end{math} den kritiske vinkel. \begin{math}n_2\end{math} er luftens brydningsindex og \begin{math}n_1\end{math} er sprits brydningsindex.
\begin{equation*}
    sin(I_c)=\frac{1}{1.36}
\end{equation*}
\begin{equation*}
    sin(I_c)=0.73529
\end{equation*}
\begin{equation*}
    I_c=sin^{-1}(0.73529)
\end{equation*}
\begin{equation*}
    \mathcolorbox{yellow}{I_c=47.3^{\circ}}
\end{equation*}
Altså vil der være totalreflektion når indfaldsvinklen er større end 47.3 grader.
\newpage