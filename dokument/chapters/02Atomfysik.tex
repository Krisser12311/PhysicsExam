\section{Atomfysik}
\subsection{Beregn energiniveauerne for skallerne n=2 og n=3}
For at finde frem til energiniveauerne for skallerne n=2 og n=3 skal man bruge den følgende formel:
\begin{equation*}
    E_n=-h\cdot c \cdot R \cdot \frac{1}{n^2}
\end{equation*}
Her er \begin{math}E_n\end{math} energiniveauet for den specefikke skal, \begin{math}h\end{math} er plancks konstant, \begin{math}c\end{math} er lysets hastighed, \begin{math}R\end{math} er Rydbergs konstant og \begin{math}n\end{math} er skallenummeret.
For at finde energiniveauet for skallen n=2 indsættes værdierne i formlen og regnes ud.
\begin{equation*}
    E_2=-6,63 \cdot 10^{-34} \frac{J}{s} \cdot 3 \cdot 10^8 m/s \cdot 1,097 \cdot 10^7 m^{-1}\cdot \frac{1}{2^2}
\end{equation*}
\begin{equation*}
    \mathcolorbox{yellow}{E_2=-5,45 \cdot 10^{-19} J}
\end{equation*}
\begin{equation*}
    E_3=-6,63 \cdot 10^{-34} frac{J}{s}  \cdot 3 \cdot 10^8 m/s \cdot 1,097 \cdot 10^7 m^{-1}\cdot \frac{1}{3^2}
\end{equation*}
\begin{equation*}
    \mathcolorbox{yellow}{E_3=-2,42 \cdot 10^{-19} J}
\end{equation*}
\subsubsection{Beregn energien for elektronovergangen fra 3 til 2}
For at finde energien for elektronovergangen fra 3 til 2 skal
\begin{equation*}
    E_{foton} = h \cdot f = E_n - E_m
\end{equation*}
Her er E\_n og E\_m energiniveauerne for de to skaller. h er plancks konstant og f er frekvensen.
\begin{equation*}
    E_{foton} = -5,45 \cdot 10^{-19} J - 2,42 \cdot 10^{-19} J
\end{equation*}
\begin{equation*}
    \mathcolorbox{yellow}{E_{foton} = -7,87 \cdot 10^{-19} J}
\end{equation*}

\subsection{Bestem frekvensen og bølgelængden af de fotoner der vil blive udsendt ved elektronovergangen}
For at finde bølgelængden af fotonen skal man bruge Rydbergs formel: 
\begin{equation*}
    \frac{1}{\lambda}=R \cdot (\frac{1}{n^2} - \frac{1}{m^2} )
\end{equation*}
Så skal lamda isoleres. 
\begin{equation*}
    \lambda = 1,097 \cdot 10^7 m^{-1} (\frac{1}{2^2} - \frac{1}{3^2} )
\end{equation*}
\begin{equation*}
    \mathcolorbox{yellow}{\lambda = 656 nm}
\end{equation*}
For at finde frekvensen kan man bruge formlen:
\begin{equation*}
    v=\frac{\lambda}{T}
\end{equation*}
Hvilket kan omskrives til 
\begin{equation*}
    v=\lambda \cdot f
\end{equation*}
Så isoleres f for at finde frekvensen.
\begin{equation*}
    f=\frac{v}{\lambda}
\end{equation*}
Så kan man indsætte værdierne i formlen og regne ud.
\begin{equation*}
    f=\frac{3 \cdot 10^8 m}{656 \cdot 10^{-9} m}
\end{equation*}
\begin{equation*}
    \mathcolorbox{yellow}{f=4,584 \cdot 10^{14} Hz}
\end{equation*}
\newpage
