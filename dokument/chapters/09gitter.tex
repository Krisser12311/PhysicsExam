\newpage
\section{Eksamensopgave 9 - Gitter}
\subsection{Bestem gitterkonstanten}
Når vi skal beregne gitterkonstanten \begin{math}d\end{math} ved vi at på 1mm er der 300 spalter, derfor er formlen
\begin{equation*}
    d=\frac{1mm}{300 spalter} = 0,00333mm = 3,33 \cdot 10^{-6} m
\end{equation*}


\subsection{Beregn afbøjningsvinklen \begin{math}\phi_{2}\end{math}}
For at beregne afbøjningsvinklen kan vi bruge denne formel.
\begin{equation*}
    sin(\phi)=\frac{n\cdot\lambda}{d}
\end{equation*}
\begin{math}\phi = vinkel\end{math} [deg]\newline
n = orden\newline
\begin{math}\lambda = \text{bølgelængde}\end{math} [m]\newline
d = gitterkonstant [m]\newline

\begin{equation*}
    sin(\phi_{2})=\frac{2\cdot 590\cdot 10^{-9}}{3,33\cdot 10^{-6}}
\end{equation*}
\begin{equation*}
    sin(\phi_{2})=0,354
\end{equation*}
\begin{equation*}
    \phi_{2}=sin^{-1}(0,354)=20,75^{\circ}
\end{equation*}
\subsection{Bestem antallet af ordner der kan ses}
For at beregne antallet af order der kas ses kan vi bruge denne formel.
\begin{equation*}
    n=\frac{d}{\lambda}
\end{equation*}
\begin{equation*}
    n=\frac{3,33\cdot 10^{-6}}{590\cdot 10^{-9}} = 5,64
\end{equation*}
Da man ikke kan se et orden med et kommatal som 0,64 skal vi runde ned til 5. Derfor er svaret at man kan se 5 ordner.

\newpage