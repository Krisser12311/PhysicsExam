\section{Varmelære}
\subsection{Beregn varmen der skal bruges til at opvarmningen af vandet når det varmes fra 10 grader til 100 grader}
For at bergne hvor meget varme der skal til for at opvarme vandet fra 10 grader til 100 grader anvendes følgende formel:
\begin{math}Q=m \cdot c \cdot \Delta T\end{math} Her er \begin{math}m\end{math} massen, \begin{math}c\end{math} den specefikke varmekapacitet og \begin{math}\Delta T\end{math} temperaturforskellen. I opgaven her antager vi at densiten er 1000 kg/m\textsuperscript{3} og volumen er 0,8 L. Og 0,8 L er derfor lig med 0,8 kg.
Værdierne indsættes i formlen:
\begin{equation*}
    Q =  0,8 kg \cdot 4184 \frac{J}{kg \cdot  ^{\circ}C} \cdot 90^{\circ}C 
\end{equation*}
Da kg går ud med kg og grader celcus går ud med grader celcus, er der kun Joule tilbage, og det var også den forventede enhed.
\begin{equation*}
    \mathcolorbox{yellow}{Q= 301248 J}
\end{equation*}

\subsection{Beregn elkedlens nyttet virkning og forklar forskællen mellem energiforbruget og varmen der er tilført vandet.}
For at finde elkedlens nyttet virkning skal man bruge følgende formel: 
\begin{equation*}
    \eta=\frac{Q_{udnyttet}}{E_{tilført}}
\end{equation*}
Dog kan man ikke bare sætte tallene ind i formlen da man har fået givet den brugte energi i kwH og ikke i Joules. 
Derfor må man omregne det til Joules:
\begin{equation*}
    1 kWh = 3,6 \cdot 10^6 J
\end{equation*}
Det vil sige at de 0,134 kWh er lig med:
\begin{equation*}
    0,134 kWh = 0,134 kWh \cdot 3,6 \cdot 10^6 J
\end{equation*}
\begin{equation*}
    \mathcolorbox{yellow}{0,134 kWh = 482400 J}
\end{equation*}
Nu kan der bare indsættes i formlen:
\begin{equation*}
    \eta=\frac{301100 J}{482400 J}\cdot 100
\end{equation*}
\begin{equation*}
    \mathcolorbox{yellow}{\eta=62\%}
\end{equation*}

Der er flere faktorer som kommer i spil angående energiforbruget og den tilførte varme, dette er ting som:
\begin{itemize}
    \item Varme tab til omgivelserne
    \begin{itemize}
        \item Varme tab til omgivelserne refererer til den energi, der går tabt fra opvarmningsprocessen til den omgivende luft og andre materialer omkring varmekilden.
    \end{itemize}
    \item Varme tab til elkedlen
    \begin{itemize}
        \item Der vil være lidt tab til elkedlen selv, da den også vil blive opvarmet.
    \end{itemize}
    \item Varme tab til ledninger
    \begin{itemize}
        \item Alle ledninger har en vis modstand, og denne modstand vil omdanne dele af energien tilført til varme fordelt på ledningen. 
    \end{itemize}
\end{itemize}

\subsection{Beregn fællestemperaturen af tekanden og vandet når der er indtrådt termisk ligevægt}
For at denne opgave kan laves skal konceptet af varmebalance bruges som siger at den varme, der afgives af 
det varme vand, er lig den varme, der optages af tekanden, da der ingen varmetab er til omgivelserne. Her 
bruges formlen:
\begin{equation*}
    T_{fælles}= \frac{m_v \cdot c_v \cdot T_v + m_k \cdot c_k \cdot T_k}{c_v \cdot m_v + c_k \cdot m_k}
\end{equation*} % todo make fælles temperatur
Den specefikke varmekapaciteten er defineret som varmekapacitet delt med massen. \begin{math}c = \frac{C}{m}\end{math} hvilket medføre at \begin{math}c = \frac{\frac{Q}{\Delta T}}{m}\end{math}. Hviæket bliver til \begin{math}c = \frac{Q}{m \cdot \Delta T}\end{math}. Enheden bliver derfor \begin{math}\frac{J}{kg \cdot ^{\circ}C}\end{math} eller \begin{math}\frac{J}{kg \cdot K}\end{math}. Da en temperatur ændring i grader celcus er det samme som en temperatur ændring i Kelvin. Kan man blot ændre sådan at man har fælles enhed. I opgaven her har man anvendt Kelvin
\begin{equation*}
    T_{fælles}= \frac{0,6 kg \cdot 4184 \frac{J}{kg \cdot  K} \cdot 373,15 K + 0,75 kg \cdot 0,800 \cdot 10^3 \frac{J}{kg \cdot K} \cdot 295,15K}{0,6 kg \cdot 4184 \frac{J}{kg \cdot K} + 0,75 kg \cdot 0,800 \cdot 10^3 \frac{J}{kg \cdot K}}     
\end{equation*}
\begin{equation*}
    \mathcolorbox{yellow}{T_{fælles}= 358,10 K}
\end{equation*}
\begin{equation*}
\mathcolorbox{yellow}{T_{fælles}= 85^{\circ}C}
\end{equation*}

	 
    
