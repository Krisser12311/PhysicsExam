\section{Varmelære}
\subsection{Beregn varmen der skal bruges til at opvarmningen af vandet når det varmes fra 10 grader til 100 grader}
For at finde den varme der kræves for at opvarme vandet fra 10 grader til 100 grader kan beskrives ved formlen: 
\begin{equation*}
    Q=m \cdot c \cdot \Delta T
\end{equation*}
Her er \begin{math}m\end{math} masssen, \begin{math}c\end{math} (\begin{math}
    \frac{J}{kg \cdot K}
\end{math}) er den specefikke varmekapacitet og $\Delta T$ er temperaturforskellen. Og Q er den varme der kræves (Joule)
For at finde massen skal man sige: 
\begin{equation*}
    m=\rho \cdot V
\end{equation*}
Her er $\rho$ densitet og V volumen.
\begin{equation*}
    m=997 \frac{kg}{m^3} \cdot 0,8 L
\end{equation*}
\begin{equation*}
    m=0,8 \cdot 10^{-3} m^3  \cdot 100 \frac{kg}{m^3}=0,08 kg
\end{equation*}
\begin{equation*}
    Q=0,8 kg \cdot 4180 \frac{J}{kg \cdot C} \cdot 90C
\end{equation*}
\begin{equation*}
    \mathcolorbox{yellow}{Q=301100 J}
\end{equation*}
\subsection{Beregn elkedlens nyttet virkning og forklar forskællen mellem energiforbruget og varmen der er tilført vandet.}
For at finde elkedlens nyttet virkning skal man bruge følgende formel: 
\begin{equation*}
    \eta=\frac{Q_{udnyttet}}{E_{tilført}}
\end{equation*}
Dog kan man ikke bare sætte tallene ind i formlen da man har fået givet den brugte energi i kwH og ikke i Joules. 
Derfor må man omregne det til Joules:
\begin{equation*}
    1 kWh=3,6 \cdot 10^6 J
\end{equation*}
Det vil sige at de 0,134 kWh er lig med:
\begin{equation*}
    0,134 kWh=0,134 kWh \cdot 3,6 \cdot 10^6 J
\end{equation*}
\begin{equation*}
    \mathcolorbox{yellow}{0,134 kWh=482400 J}
\end{equation*}
Nu kan der bare indsættes i formlen:
\begin{equation*}
    \eta=\frac{301100 J}{482400 J}\cdot 100
\end{equation*}
\begin{equation*}
    \mathcolorbox{yellow}{\eta=62\%}
\end{equation*}

\subsection{Beregn fællestemperaturen af tekanden og vandet når der er indtrådt termisk ligevægt}
For at finde fællestemperaturen skal man bruge følgende formel:
\begin{equation*}
    T_{fælles}= \frac{m_a \cdot C_a \cdot T_a + m_v c_v \cdot T_v}{m_g}
\end{equation*} % todo make fælles temperatur

\newpage