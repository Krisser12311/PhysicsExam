\newpage
\section{Eksamensopgave 7 - Gasser}
\subsection{Beregn tyngdekraften på ballon uden luft}
Først start med at skriver man de kendte værdier:
\\\\
Volume: \begin{math}3m^3\end{math} \newline
Vægt: \begin{math}0,110kg\end{math} \newline
Ved brug af newtons 2. lov \begin{math}F_{res}=m\cdot a\end{math} kan man udregne tyngdekraften på ballonen uden luft. \newline
Jordens tyngdeacceleration er ca. \begin{math}9,82 m/s^2\end{math} som kan også skrives som \begin{math}9,82N/kg\end{math} det varierer lidt i forhold til hvor på jorden man er. \newline

\begin{equation}
	F_t=0,110kg\cdot9,82 N/kg=1,08N
\end{equation}
\subsection{Skitser kræfterne der påvirker ballonen og beregn opdriften. Det kan antaages at tempraturen er 20 grader C og at densiteten for luften i lokalet dermed er \begin{math}\rho=1,20kg/m^3\end{math}}
Igen begynder vi med at skrive de brugbar værdier ned:
\\\\
Densiteten: \begin{math}1,20kg/m^3\end{math} \newline
Volume: \begin{math}3m^3\end{math} \newline
Ved brug af Archimedes' lov \begin{math}F_{op}=\rho\cdot V\cdot g\end{math}, kan man udregne opdriften der påvirker ballonen, tyngdeaccelerationen er stadig \begin{math}9,82 N/kg\end{math}
\begin{equation}
	F_{op}=1,20kg/m^3 \cdot 3m^3 \cdot 9,82N/kg=3,535N
\end{equation}

