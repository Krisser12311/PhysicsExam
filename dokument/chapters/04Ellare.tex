\newpage
\section{Eksamensopgave 4 - Ellære}
\subsection{Beregn erstatningsresistansen i resistorkoblingen}
Når man beregner erstatningsresistansen i en resistorkobling, ønsker man at finde én enkelt resistor, som kan erstatte hele koblingen og have samme samlede modstand. Kan man starte med at finde de resistorer der sidder i parallelforbindelsen, og beregne deres erstatningsresistansen. 
I denne opgave skal i beregne modstanden for en parallel forbindelse som også har en serie forbindelse. Dog inden man gør det angiver man hvad man vil kalde ens modstande for ikke at der bilver forvirrene.
\begin{figure}[h!]
    \centering
    \includegraphics[width=0.5\textwidth]{figures/resistans.png}
    \caption{Indtegning af navne samt resistorkobling}
\end{figure}
Herefter skal man finde frem til værdien for den samlede erstatningsresistansen, da det er i en parallelforbindelsen anvendes formlen for en parallelforbindelse.
\begin{equation*} % TODO: Evt nævn at det er kirchhoffs 1. lov
    R_{12}=R_{1}+R_{2}
\end{equation*}
\begin{equation*}
    \mathcolorbox{yellow}{R=5\Omega+21\Omega=26\Omega}
\end{equation*}
Formlen for en parallel forbindelse ser sådan ud:
\begin{equation*}
    \frac{1}{R_{total}}=\frac{1}{R_{1}}+\frac{1}{R_{2}}
\end{equation*}
Da de er parallel skal man huske at tage den reciprokke værdi til sidst.
\begin{equation*}
    R_{total}=\left(\frac{1}{26\Omega}+\frac{1}{120\Omega}\right)^{-1}
\end{equation*}

\begin{equation*}
    \frac{1}{R_{total}}=\frac{1}{26\Omega}+\frac{1}{120\Omega}=0,04679\Omega^{-1}=\mathcolorbox{yellow}{21,37\Omega}
\end{equation*}

\subsection{Modstanden på 5~$\Omega$ laves af en kobbertråd med længden 60m. Bestem trådens radius.}
For at beregne kobbertrådens radius kan vi bruge følgende formel:
\begin{equation*}
    R=\rho\cdot\frac{l}{A}
\end{equation*}
R = Modstand i ohm ~$\left[ \Omega \right]$\newline
    $\rho$ = Resistivitet for kobbertråden er den på $0,0155 \cdot 10^{-6} \Omega$ \newline
    $l$ = Længden af kobbertråden\newline
    $A$ = Tværsnitsarealet\newline

For at bruge denne formel skal vi have isoleret A og derfor bliver formlen til.
\begin{equation*}
    A=\frac{l\cdot\rho}{R}
\end{equation*}
Nu kan vi indsætte vores værdier
\begin{equation*}
    A=\frac{60m\cdot0,0155 \cdot 10^{-6} \Omega \cdot m}{5\Omega}
\end{equation*}
\begin{equation*}
    A = \frac{9,3 \cdot 10^{-7} \cdot \Omega \cdot m^{2}}{5 \Omega} = 1,86 \cdot 10^{-7} \cdot m^{2} \approx 0,186 \cdot mm^{2}
\end{equation*}
Man har nu tværsnitsarealet og kan nu beregne radiusen af tråden ved bruge af formlen for aralet af en cirkel.
\begin{equation*}
    A=\pi*r^{2}
\end{equation*}
For at få radiusen isoleret skal vi isolere r. Dette gør man ved at dividere med $\pi$.
\begin{equation*}
    r^{2}=\frac{A}{\pi}
\end{equation*}
Hvorefter man tager kvadratroden af dette for at finde radiusen.
\begin{equation*}
    r=\sqrt{\frac{A}{\pi}}
\end{equation*}
Værdierne kan nu indsættes.
\begin{equation*}
    r=\sqrt{\frac{0,186mm^{2}}{\pi}} = \mathcolorbox{yellow}{0,243 mm}
\end{equation*}

\subsection{Et batteri med hvilespændingen 1,4V og en indre modstand på 0,5~$\Omega$ tilsluttes resistorkoblingen. Skitser situationen og beregn polspændingen.}
\begin{equation*}
    U_{pol}=U_{hvile}-I\cdot R_{indre}
\end{equation*}
\begin{equation*}
    I_{total}=\frac{1,4V}{21,37\Omega+0,5\Omega}=0,064A
\end{equation*}

\begin{equation*}
    U_{pol}=1,4V-0,5\Omega\cdot 0,064A=1,368V
\end{equation*}
\newpage